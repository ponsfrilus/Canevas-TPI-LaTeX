\chapter{Analyse / Conception}
\section{Concept}

Le concept complet avec toutes ses annexes. Par exemple : 
\begin{itemize}
  \item Multimédia: carte de site, maquettes papier, story board préliminaire, …
  \item Bases de données: interfaces graphiques, modèle conceptuel.
  \item Programmation: interfaces graphiques, maquettes, analyse fonctionnelle…
  \item …
\end{itemize}

\section{Stratégie de test}

Décrire la stratégie globale de test :
\begin{itemize}
  \item types de des tests et ordre dans lequel ils seront effectués.
  \item les moyens à mettre en œuvre.
  \item couverture des tests (tests exhaustifs ou non, si non, pourquoi ?).
  \item données de test à prévoir (données réelles ?).
  \item les testeurs extérieurs éventuels.
\end{itemize}

\section{Risques techniques}
\begin{itemize}
  \item risques techniques (complexité, manque de compétences, …).
\end{itemize}
Décrire aussi quelles solutions ont été appliquées pour réduire les risques (priorités, formation, actions, …).

\section{Planification}

Révision de la planification initiale du projet :
\begin{itemize}
  \item planning indiquant les dates de début et de fin du projet ainsi que le découpage connu des diverses phases. 
  \item partage des tâches en cas de travail à plusieurs.
\end{itemize}
Il s’agit en principe de la planification définitive du projet. Elle peut être ensuite affinée (découpage des tâches). Si les délais doivent être ensuite modifiés, le responsable de projet doit être avisé, et les raisons doivent être expliquées dans l’historique.

\section{Dossier de conception}

Fournir tous les document de conception :
\begin{itemize}
  \item le choix du matériel HW
    \item le choix des systèmes d'exploitation pour la réalisation et l'utilisation
    \item le choix des outils logiciels pour la réalisation et l'utilisation
    \item site web: réaliser les maquettes avec un logiciel, décrire toutes les animations sur papier, définir les mots-clés, choisir une formule d'hébergement, définir la méthode de mise à jour, …
    \item bases de données: décrire le modèle relationnel, le contenu détaillé des tables (caractéristiques de chaque champs) et les requêtes.
    \item programmation et scripts: organigramme, architecture du programme, découpage modulaire, entrées-sorties des modules, pseudo-code / structogramme…
\end{itemize}

Le dossier de conception devrait permettre de sous-traiter la réalisation du projet !
